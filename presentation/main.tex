\documentclass{beamer}

\mode<presentation>
{
  \usetheme{Madrid}      % or try Darmstadt, Madrid, Warsaw, ...
  \usecolortheme{default} % or try albatross, beaver, crane, ...
  \usefonttheme{professionalfonts}  % or try serif, structurebold, ...
  \setbeamertemplate{navigation symbols}{}
  \setbeamertemplate{caption}[numbered]
} 

\usepackage[english]{babel}
\usepackage[utf8]{inputenc}
\usepackage[T1]{fontenc}

\title[presentation]{DESCRIPTION OF THE WOBBLING MOTION THROUGH A BOSON METHOD}
\author{Robert Poenaru}
\institute{DFT, IFIN-HH\\Doctoral School of Physics, UB}
\date{\today}

\begin{document}

\begin{frame}
  \titlepage
\end{frame}

\begin{frame}{Outline}
 \tableofcontents
\end{frame}

\section{Introduction}



\begin{frame}{Tables and Figures}

\begin{itemize}
\item Use \texttt{tabular} for basic tables --- see Table~\ref{tab:widgets}, for example.
\item You can upload a figure (JPEG, PNG or PDF) using the files menu. 
\item To include it in your document, use the \texttt{includegraphics} command (see the comment below in the source code).
\end{itemize}

% Commands to include a figure:
%\begin{figure}
%\includegraphics[width=\textwidth]{your-figure's-file-name}
%\caption{\label{fig:your-figure}Caption goes here.}
%\end{figure}

\begin{table}
\centering
\begin{tabular}{l|r}
Item & Quantity \\\hline
Widgets & 42 \\
Gadgets & 13
\end{tabular}
\caption{\label{tab:widgets}An example table.}
\end{table}

\end{frame}



\end{document}
